 
\documentclass[10pt]{article}         %% What type of document you're writing.
\usepackage{graphicx}
\usepackage{hyperref}
\usepackage[dvipsnames]{xcolor}

%%%%% Preamble

%% Packages to use

\usepackage{amsmath,amsfonts,amssymb}   %% AMS mathematics macros

%% Title Information.

\title{Reporte tecnico}
\author{Denisse Arely Gonzalez Santos}
%% \date{2 July 2004}           %% By default, LaTeX uses the current date

%%%%% The Document

\begin{document}

\maketitle


\section{Introduccion}

El presente trabajo es un reporte tecnico del desarrollo de una red neuronal,antes de empezar definiremos que es una red neuronal artificial y su objetivo.
\\Las redes neuronales artificiales (también conocidas como sistemas conexionistas) son un modelo computacional el que fue evolucionando a partir de diversas aportaciones científicas que están registradas en la historia.1 Consiste en un conjunto de unidades, llamadas neuronas artificiales, conectadas entre sí para transmitirse señales. La información de entrada atraviesa la red neuronal (donde se somete a diversas operaciones) produciendo unos valores de salida.

El objetivo de la red neuronal es resolver los problemas de la misma manera que el cerebro humano, aunque las redes neuronales son más abstractas. Las redes neuronales actuales suelen contener desde unos miles a unos pocos millones de unidades neuronales.

Ademas comprender el proceso de toma de decisiones y cómo las tecnologías de información, en
especial la inteligencia artificial con el manejo de redes neuronales, pueden servir de apoyo
a dicho proceso, es el objetivo central de este trabajo de investigación, el cual se ha
actualizado y ha permitido revisar los resultados de la red neuronal.

\section{Desarrollo}

De aqui en adelante empezaremos con el desarrollo de la red neuronal, mi tema para este proyecto sera simbolos griegos que en total serian en total 18 simbolos los que usare.
\section{Conclusion}


\end{document}